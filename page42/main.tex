\documentclass{article}

% if you need to pass options to natbib, use, e.g.:
%     \PassOptionsToPackage{numbers, compress}{natbib}
% before loading neurips_2019

% ready for submission
% \usepackage{neurips_2019}

% to compile a preprint version, e.g., for submission to arXiv, add add the
% [preprint] option:
    % \usepackage[preprint]{neurips_2019}

% to compile a camera-ready version, add the [final] option, e.g.:
\usepackage[final]{neurips}

% to avoid loading the natbib package, add option nonatbib:
    % \usepackage[nonatbib]{neurips_2019}

\usepackage[utf8]{inputenc} % allow utf-8 input
\usepackage[T1]{fontenc}    % use 8-bit T1 fonts
\usepackage{hyperref}       % hyperlinks
\usepackage{url}            % simple URL typesetting
\usepackage{booktabs}       % professional-quality tables
\usepackage{amsfonts}       % blackboard math symbols
\usepackage{nicefrac}       % compact symbols for 1/2, etc.
\usepackage{microtype}      % microtypography
\usepackage{graphicx}
\usepackage{tikz}
\usepackage{amsmath}

\usepackage{xepersian}
\settextfont{XB Yas.ttf}

\title{Direct Search Method}



% The \author macro works with any number of authors. There are two commands
% used to separate the names and addresses of multiple authors: \And and \AND.
%
% Using \And between authors leaves it to LaTeX to determine where to break the
% lines. Using \AND forces a line break at that point. So, if LaTeX puts 3 of 4
% authors names on the first line, and the last on the second line, try using
% \AND instead of \And before the third author name.


\author{%

   نرم افزار ریاضی\\

  \texttt{@gmail.com} \\
}



\begin{document}
\baselineskip=0.7cm

\begin{minipage}{0.1\textwidth}% adapt widths of minipages to your needs
\includegraphics[width=1.2cm]{Amirkabir.jpg}
\end{minipage}%
\hfill%
\begin{minipage}{0.9\textwidth}\raggedleft
دانشگاه صنعتی امیرکبیر\\
مباحث ویژه در بهینه سازی\\
\end{minipage}
% \end{}


\makepertitle



\section*{مسایل بهینه سازی نا مقید یک متغیره:.}
\begin{flushleft}
$min \hspace{3mm}f(x)$\\
$x \in \mathbb{R}$
\end{flushleft}



\section*{  تعریف 
function uni-module  :}
تابع 
$f(x): \mathbb{R}\rightarrow \mathbb{R}$
را در بازه [a,b]
را یک تابع unimodule
گوییم هر گاه اولا تابع در این بازه تنها یک مینیمم داشته باشد و ثانیا اگر
$x_1<x_2<x^*$
آنگاه $f(x_1)>f(x_2)$\\
و اگر $x^*<x_1<x_2$
آنگاه 
$f(x_1)<f(x_2)$  \\
نمونه ها:\\
\begin{flushleft}
هموار
\begin{tikzpicture}
\draw[->] (-3,0) -- (3,0) ;
\draw[->] (0,-3) -- (0,3) ;

\draw[ultra thick,gray,domain=-1:2,smooth] plot (\x,{(\x)^2-(\x)+1}) ;
\filldraw [red] (0,1) circle (2pt);
\filldraw [red] (1,1) circle (2pt);
\filldraw [red] (1.5,1.75) circle (2pt);
\end{tikzpicture}

\end{flushleft}
\begin{flushleft}
ناهموار
\begin{tikzpicture}
\draw[->] (-3,0) -- (3,0) ;
\draw[->] (0,-3) -- (0,3) ;
\draw (-3, 3) to[out=0, in=135] (1, 2);
\draw (1, 2) to[out=45, in=180] (3, 3);

\end{tikzpicture}

    
\end{flushleft}
\begin{flushleft}
ناپیوسته
\begin{tikzpicture}
\draw[->] (-3,0) -- (3,0) ;
\draw[->] (0,-3) -- (0,3) ;
\draw (-3, 3) -- (-2, 2);
\draw (1,2) -- (0,3);
\draw (2, 2) -- (3, 3);
\draw (1,-1) -- (2,0);
\filldraw [red] (-2,2) circle (2pt);
\filldraw [red] (3,3) circle (2pt);
\filldraw [red] (-2,2) circle (2pt);
%\draw[fill] (0,0)-- (90:#1) arc (90:270:#1) -- cycle 
\draw[orange, ultra thick] (2,0) circle (3pt);
\draw[orange, ultra thick] (0,3) circle (3pt);
\end{tikzpicture}

    
\end{flushleft}
\begin{flushleft}
unimdle نیست
\begin{tikzpicture}
\draw[->] (0,0) -- (6,0) ;
\draw[->] (0,0) -- (0,6) ;
    \draw (1, 7.5) to[out=-90, in=90] (1, 7) to[out=-45, in=180] (2, 6) to[out=0, in=-135] (2, 6);
  \draw (2, 6) to[out=0, in=90] (2, 5.5) to[out=-45, in=90] (2, 5.5) to[out=-45, in=90] (2.5, 5.5) to[out=-90, in=90] (2.5, 4.5) to[out=-90, in=90] (2.5, 4.5) to[out=-45, in=180] (3, 3.5) to[out=45, in=-90] (3, 3.5) to[out=45, in=180] (3, 3.5) to[out=0, in=-90] (3.5, 5) to[out=90, in=-90] (3.5, 5.5);
  \draw (3.5, 5.5) to[out=90, in=-90] (3.5, 5.5) to[out=45, in=-90] (4, 5.5) to[out=90, in=180] (5.5, 7);
\end{tikzpicture}
\end{flushleft}
\pagebreak
هدف یافتن نقطه مینیمم تابع f
در بازه [a,b]
است به طوری که تابع f
در بازه [a,b]
unimodule است 
\\
وجود دارد، از جمله 

\begin{enumerate}
\item fixed step search method
\item Exhuastive search method
\item dichotomous search method
\item Fibonnaci search method
\item Golden section ratio method
\item Interpolation based search
\item ....
\end{enumerate}
\end{document}
