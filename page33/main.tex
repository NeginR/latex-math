\documentclass{article}
\usepackage{fullpage}
\usepackage[final]{neurips}
\usepackage[utf8]{inputenc} % allow utf-8 input
\usepackage{titlesec}
\usepackage{diagbox}
\usepackage{adjustbox}
\usepackage{graphicx}
\usepackage{xepersian}
\settextfont{XB Yas.ttf}

\title{صفحه 33}

\author{%

  نرم افزار ریاضی\\
  \texttt{@gmail.com} \\
}



\begin{document}
\baselineskip=0.7cm

\begin{minipage}{0.1\textwidth}% adapt widths of minipages to your needs
\includegraphics[width=1.2cm]{Amirkabir.jpg}
\end{minipage}%
\hfill%
\begin{minipage}{0.9\textwidth}\raggedleft
دانشگاه صنعتی امیرکبیر\\
مباحث ویژه در بهینه سازی\\
\end{minipage}

\makepertitle

مقایسه کلی بین روش های خطی:\\

\begin{enumerate}
\item جستجوی خطی با حهت تند ترین کاهش
\item جستجوی خطی با جهت نیوتونی
\item جستجوی خطی با جهت شبه نیوتونی
\item ناحیه اطمینان با جهت شبه نیوتونی
\item روش CG\\\\
\end{enumerate}

\begin{flushleft}

\begin{tabular}{|l|*{12}{c|}|}\hline

\slashbox{روش}{فاکتور}
&\makebox[5em]{همگرایی سراسری}&\makebox[4em]{مرتبه همگرایی}&\makebox[3em]{اعتمادپذیری}
&\makebox[7em]{هرتکرارحجم محاسباتی}&\makebox[6em]{فضای محاسباتی هرتکرار}&\makebox[5em]{پیاده سازی آسان}\\\hline\hline

LSباجهت تندترین جهت
&1&1&1&6&2&5\\\hline
LSباجهت نیوتونی
&1&3&2&1&1&1\\
\hline
LSباجهت شبه نیوتونی
&1&2&3&2&1&3\\\hline
TRباجهت نیوتونی
&5.1&3&2&1&1&1\\\hline
TRباجهت شبه نیوتونی
&5.1&2&6&2&1&2\\\hline
CG
&1&?&3&3&4&5\\\hline


\end{tabular}

\end{flushleft}

\end{document}


