\documentclass{article}

\usepackage{bbding}
\usepackage[final]{neurips}
\usepackage{amsmath}
\usepackage{mathtools}
\usepackage[utf8]{inputenc} % allow utf-8 input

\usepackage{amsfonts}       % blackboard math symbols

\usepackage{graphicx}
\usepackage{xepersian}
\settextfont{XB Yas.ttf}

\title{صفحه 58}

\author{%
  
  نرم افزار ریاضی\\

  \texttt{@gmail.com} \\
}



\begin{document}
\baselineskip=0.7cm

\begin{minipage}{0.1\textwidth}% adapt widths of minipages to your needs
\includegraphics[width=1.2cm]{Amirkabir.jpg}
\end{minipage}%
\hfill%
\begin{minipage}{0.9\textwidth}\raggedleft
دانشگاه صنعتی امیرکبیر\\
مباحث ویژه در بهینه سازی\\
\end{minipage}
% \end{}


\makepertitle


بنابراین:\\

\[\begin{bmatrix*}
X^T & \lambda^T
\end{bmatrix*}
\times
\begin{bmatrix*}
H\lambda+A^T\lambda\\
AX
\end{bmatrix*}
 = 0\]
\[X^THX+\underbrace{X^TA^T}_\text{$X^TA^T=0$}\lambda+\lambda^T\underbrace{AX}_\text{$AX=0$}=0   \Rightarrow   X^THX=0\]\\
از طرفی
$X=Zu$
و لذا
$U^TZ^THZU=0$ 
حال چون طبق فرض قضیه، 
$ Z^THZ $ 
ماتریس صفر مثبت است پس نتیجه می گیریم که باید 
$U=0$
و از اینکه 
$X=Z=0 $ 
نتیجه می گیریم که    
$X=0$ 
و از رابطه 
$HX+A^T\lambda=0$ 
و با توجه به اینکه $A$
مرتبه سطری کامل است نتیجه می گیریم که $\lambda=0 $
، پس از فرض
$\begin{bmatrix*}
 H & A^T \\
 A & 0
\end{bmatrix*}
\times
\begin{bmatrix*}
X\\
\lambda
\end{bmatrix*}
 = \begin{bmatrix*}
0\\
0
\end{bmatrix*}$
نتیجه گرفتیم که
$\begin{bmatrix*}
 X\\
 \lambda
\end{bmatrix*}=
\begin{bmatrix*}
0\\
0
\end{bmatrix*}$
که ثابت می شود، دستگاه جواب منحصر به فرد دارد.\\
\begin{flushleft}
\raggedright{\RectangleBold\quad}\\
\end{flushleft}
\baselineskip=0.7cm
قضیه: در مساله(*) اگر 
$Z$ماتریس پوچی $A$
باشد و $Z^THZ$
ماتریس معین مثبت باشد ، آنگاه جواب حاصل از حل دستگاه 
$KKT$ (که جوابی منحصر بفرد است.)
جواب مینیمم سراسری و اکید برای مساله (*)
می باشد. در حالیکه $Z^THZ$
معین مثبت نباشد، امکان دارد دستگاه جواب داشته باشدویا نداشته باشد.\\
نکته:در مسایل (*)می توان با حذف قیود تساوی ، مساله را به یک مساله نامقید و کوچک تر تبدیل کرد.\\
در ادامه این نکته توضیح داده می شود:\\
قیود $AX=B$
را در نظر بگیرید و فرض کنید $Z $
ماتریس پوچی $A$
باشد.($AZ=0$) و  
$X_0$
یک جواب دلخواه مساله است.
\end{document}
